% Type Soundness of Amal's SoupPCF
% Author: Hyeyoung Shin
% Date: 7 Feb 2018
% Updated: 7 Feb 2018
% Copyright: William DeMeo 2014

\documentclass{article}
%% If you want bigger font, replace the line above with
%%     \documentclass[12pt]{article}  


%%%%%%%%%%%%%%%%%%%%%%%%%%%%%%%%%%%%%%%%
% Basic packages
%%%%%%%%%%%%%%%%%%%%%%%%%%%%%%%%%%%%%%%%
\usepackage{amsmath,amsthm,amssymb,mathtools}
\usepackage{etoolbox}
\usepackage{mathpartir}
\usepackage{xcolor}
\usepackage{xspace}
\usepackage{url}



%%%%%%%%%%%%%%%%%%%%%%%%%%%%%%%%%%%%%%%%
% Acronyms -- add your own if you want!
%%%%%%%%%%%%%%%%%%%%%%%%%%%%%%%%%%%%%%%%
\usepackage[acronym, shortcuts]{glossaries}
\newacronym{HoTT}{HoTT}{homotopy type theory}
\newacronym{IPL}{IPL}{intuitionistic propositional logic}
\newacronym{TT}{TT}{intuitionistic type theory}
\newacronym{LEM}{LEM}{law of the excluded middle}
\newacronym{ITT}{ITT}{intensional type theory}
\newacronym{ETT}{ETT}{extensional type theory}
\newacronym{NNO}{NNO}{natural numbers object}
\robustify{\ac}
%% Use acronyms like this: \ac{IPL}
%% If the acronym has already been used, then \ac{IPL} just prints IPL; 
%% However, if it's the first occurrence, then we'll get:
%%   intuitionistic propositional logic (IPL)


\newrobustcmd*{\vocab}[1]{\emph{#1}}
\newrobustcmd*{\latin}[1]{\textit{#1}}


%%%%%%%%%%%%%%%%%%%%%%%%%%%%%%%%%%%%%%%%%%%%%%%%%%%%%%%%%%%%%%%%%%%
% `comment` is a package that enables you to omit sections by simply
% enclosing the unwanted stuff inside `\begin{comment}  \end{comment}`
\usepackage{comment}  % This package lets you omit sections of code


%%%%%%%%%%%%%%%%%%%%%%%%%%%%%%%%%%%%%%%%%%%%%%%%%%%%%%%%%%%%%%%%%%%
% `hyperref` is a package that gives you hyperlinks in your pdf,
% which can be very helpful for those reading it on a computer
% or tablet.  You get the default color options for links with
%
%          \usepackage{hyperref}
%
% but I think the default colors are VERY ugly and distracting,
% so I use the following options:
\usepackage[
  colorlinks=true,
  urlcolor=black,
  linkcolor=black,
  citecolor=black
]{hyperref}


%%%%%%%%%%%%%%%%%%%%%%%%%%%%%%%%%%%%%%%%%%%%%%%%%%%%%%%%%%%%%%%%%%%%
%
% The `geometry.sty` package is for customizing page layout
%
\usepackage[  
  top    = 3cm,  %% <<== adjust top margin by changing this number
  left   = 4cm,  %% <<== adjust left margin by changing this number
  bottom = 3cm,  %% <<==  ...etc...
  right  = 4cm
]{geometry}


%% The next line incorporates some macros employed by Bob's students.
\usepackage{cmu-macros}  %% see the `cmu-macros.sty` file for details.


%%%%%%%%%%%%%%%%%%%%%%%%%%%%%%%%%%%%%%%%%%%%%%%%%%%%%%%%%%%%%%%%%%%%
%
% The `proof-dashed.sty` package is for typesetting inference rules
%
\usepackage{proof-dashed}
%
% Look at the comments in the file `proof-dashed.sty` for more details
% about how to use it, but looking at this one example may be sufficient.
% (Notice you start with the base of the tree and work up.)
%
% Example: The LaTeX commands
%
%		\infer[R2]{ E }
%                     { A & \infer[R1]{ D }
%                                   { B & C } }
%
%                produce following derivation tree
%
%                       B C
%                      ----- R1
%                  A     D
%                 ---------- R2
%                     E
%
% Note: spacing doesn't matter, so you could write the above derivation
% like this:  \infer[R2]{E}{A & \infer[R1]{D}{B & C}}


%%%%%%%%%%%%%%%%%%%%%%%%%%%%%%%%%%%%%%%%%%%%%%%%%%%%%%%%%%%%%%%%%%%%
% `tikz` is a package for making diagrams
% `tikz-cd` has additional support for commutative diagrams
%  We don't need it for this document, but if you will add diagrams,
%  be sure you have the `tikz-cd.sty` file in latex search path,
%  and then uncomment the next line.
%% \usepackage{tikz-cd}  


%%%%%%%%%%%%%%%%%%%%%%%%%%%%%%%%%%%%%%%%%%%%%%%%%%%%%%%%%%%%%%%%%%%%
% `fancy` is a package for making ``fancy'' headers and footers.
%
% If you want fancy headers and footers, then uncomment the next three lines.
%
%%   \pagestyle{fancy}
%%   \lhead{Democrat}   \chead{Moderate}   \rhead{Republican}
%%   \lfoot{}  \cfoot{\thepage}   \rfoot{}
%    \renewcommand{\headrulewidth}{0.4pt}



%%%%%%%%%%%%%%%%%%%%%%%%%%%%%%%%%%%%%%%%%%%%%%%%%%%%%%%%%%%%%%%%%%%%%%%%%%%%%%%
%=============================================================================%
%%%%%%%%%%%%%%%%%%%%%%%%%%%%%%%%%%%%%%%%%%%%%%%%%%%%%%%%%%%%%%%%%%%%%%%%%%%%%%%

\title{Amal's SoupPCF}
\author{Hyeyoung Shin\\
  \url{shin.hy@husky.neu.edu}}
\date{7 Feb 2018}


\begin{document}  %% <<== Here's where the document actually begins!

\maketitle

\section{Syntax}

$\begin{array}{l c l}
  t    & ::= & unit \mid int \mid t \rightarrow t \mid LS\\[1em]
  e    & ::= & () \mid n \mid if0~e_1~e_2~e_3 \mid e_1~p~e_2 \mid x \mid lam~x:t.e \mid e_1~e_2 \mid nil \mid consl~e_1~e_2 \mid conss~e\\[1em]
       &     &  \mid case~e~of~nil \Rightarrow e_1; consl~x~rx \Rightarrow e_2; conss~x \Rightarrow e_3\\[1em]
  p    & ::= & + \mid -\\[1em]
  \Gamma & ::= & . \mid \Gamma,x:t\\[2em]
\end{array}$


\section{Typing Judgement}

\begin{itemize}
\item[] 
\begin{mathpar}
  \infer[\mathsf{\text{T-UNIT}}]{\Gamma \vdash () : unit}{}
  \and
  \infer[\mathsf{\text{T-INT}}]{\Gamma \vdash n : int}{}
  \and
  \infer[\mathsf{\text{T-VAR}}]{\Gamma \vdash x : T}{ 
 x : T \in \Gamma} 
\end{mathpar}

\item[] 
\begin{equation*}
  \infer[\mathsf{\text{T-IF0}}]{\Gamma \vdash if0~e_1~e_2~e_3 : T}{ 
 \Gamma \vdash e_1 : int & \Gamma \vdash e_2 : T & \Gamma \vdash e_3 : T}
\end{equation*}

\item[] 
\begin{equation*}
  \infer[\mathsf{\text{T-APP}}]{\Gamma \vdash e_1~e_2 : T_2}{ 
 \Gamma \vdash e_1 : T_1 \rightarrow T_2 & \Gamma \vdash e_2 : T_1}
\end{equation*}

\item[] 
\begin{equation*}
  \infer[\mathsf{\text{T-ABS}}]{\Gamma \vdash lam~x:T_1.e : T_2}{ 
 \Gamma, x:T_1 \vdash e_2 : T_2}
\end{equation*}

\item[] 
\begin{equation*}
  \infer[\mathsf{\text{T-P}}]{\Gamma \vdash e_1~p~e_2 : int}{ 
 \Gamma \vdash e_1 : int & \Gamma \vdash e_2 : int}
\end{equation*}

\item[] 
\begin{equation*}
 \infer[\mathsf{\text{T-NIL}}]{\Gamma \vdash nil : LS}{}
\end{equation*}

\item[]
\begin{mathpar}  
  \infer[\mathsf{\text{T-CONSL}}]{\Gamma \vdash consl~e_1~e_2 : LS}{
    \Gamma \vdash e_1 : \text{int} & \Gamma \vdash e_2 : LS}
  \and
  \infer[\mathsf{\text{T-CONSS}}]{\Gamma \vdash conss~e : \text{LS}}{
    \Gamma \vdash e : unit \rightarrow LS}
\end{mathpar}

\item[]
\begin{equation*}
  \infer[\mathsf{\text{ T-CASE }}]{\Gamma \vdash \text{ case } e \text{ of nil } \Rightarrow e_1; \text{ cons-l } x \text{ } rx \Rightarrow e_2; \text{ cons-s } x \Rightarrow e_3 : t}{
    \Gamma \vdash e : \text{ LS } & \Gamma \vdash e_1 : t & \Gamma, x:\text{ int}, \text{ }rx:\text{ LS } \vdash e_2 : t & \Gamma, x:\text{ unit } \rightarrow \text{ LS } \vdash e_3 : t}
\end{equation*}
\end{itemize}


%% \newpage
\section{CBV Operational Semantics}

$\begin{array}{l c l}
  v    & ::= & () \mid n \mid lam~x:t.e \mid nil \mid consl~v_1~v_2 \mid conss~v\\[1em]
  E    & ::= & [.] \mid if0~E~e_2~e_3 \mid E~p~e_2 \mid v_1~p~E \mid E~e_2 \mid v_1~E \mid consl~E~e_2 \mid consl~v~E \mid conss~E \mid\\[1em] 
       &     &  \mid case~E~of~nil \Rightarrow e_1; consl~x~rx \Rightarrow e_2; conss~x \Rightarrow e_3\\[1em]
\end{array}$

\section{LEMMA [INVERSION]}

\begin{enumerate}
  \item If $\Gamma \vdash () : R$, then $R = uni$.
  \item If $\Gamma \vdash n : R$, then $R = int$.
  \item If $\Gamma \vdash e_1~p~e_2 : R$, then $R = int$ and $\Gamma \vdash e_1 : int$ and $\Gamma \vdash e_2 : int$.
  \item If $\Gamma \vdash x : R$, then $x : R \in \Gamma$.
  \item If $\Gamma \vdash \lambda x : T_1, e : R$, then $R = T_1 \rightarrow T_2$ for some $T_2$ with $\Gamma, x : T_1 \vdash e : T_2$.
  \item If $\Gamma \vdash e_1e_2 : R$, then $\exists~T_1$ such that $\Gamma \vdash e_1 : T_1 \rightarrow R$ and $\Gamma \vdash e_2 : T_1$.
  \item If $\Gamma \vdash nil : R$, then $R = LS$.
  \item If $\Gamma \vdash consl~e_1~e_2 : R$, then $R = LS$ and $\Gamma \vdash e_1 : int$ and $\Gamma \vdash e : LS$.
  \item If $\Gamma \vdash conss~e : R$, then $R = LS$ and $\Gamma \vdash e : unit \rightarrow LS$.
  \item If $\Gamma \vdash case~e~of~nil \Rightarrow e_1; consl~x~rx \Rightarrow e_2; conss~x \Rightarrow e_3 : R$, then $\Gamma \vdash e : LS$.\\ Besides, $\Gamma \vdash e_1 : R$, $\Gamma, x : int, rx : LS \vdash e_2: R$, and $\Gamma, x : unit \rightarrow LS \vdash e_3 : R$.
  \item If $\Gamma \vdash if0~e_1~e_2~e_3 : R$, then $\Gamma \vdash e_1 : Bool$, $\Gamma \vdash e_2 : R$, and $\Gamma \vdash e_3 : R$.
\end{enumerate}

\begin{proof}
  Immediate from the typing relations.
\end{proof}


\section{THEOREM [UNIQUENESS OF TYPES]}
In a given typing context, $\Gamma$, a term $e$ (with all free variables in $\Gamma$) has at most one type.

\begin{proof} (structural induction on term $e$)
  \begin{itemize}
  \item Case $e = ()$:\\
    $e$ has the unique unit type by INVERSION 1.
    
  \end{itemize}
\end{proof}

  

\begin{itemize}
\item[] 
\begin{mathpar}
  \infer[\mathsf{\text{T-UNIT}}]{\Gamma \vdash () : unit}{}
  \and
  \infer[\mathsf{\text{T-INT}}]{\Gamma \vdash n : int}{}
  \and
  \infer[\mathsf{\text{T-VAR}}]{\Gamma \vdash x : T}{ 
 x : T \in \Gamma} 
\end{mathpar}

\item[] 
\begin{equation*}
  \infer[\mathsf{\text{T-IF0}}]{\Gamma \vdash if0~e_1~e_2~e_3 : T}{ 
 \Gamma \vdash e_1 : int & \Gamma \vdash e_2 : T & \Gamma \vdash e_3 : T}
\end{equation*}

\item[] 
\begin{equation*}
  \infer[\mathsf{\text{T-APP}}]{\Gamma \vdash e_1~e_2 : T_2}{ 
 \Gamma \vdash e_1 : T_1 \rightarrow T_2 & \Gamma \vdash e_2 : T_1}
\end{equation*}

\item[] 
\begin{equation*}
  \infer[\mathsf{\text{T-ABS}}]{\Gamma \vdash lam~x:T_1.e : T_2}{ 
 \Gamma, x:T_1 \vdash e_2 : T_2}
\end{equation*}

\item[] 
\begin{equation*}
  \infer[\mathsf{\text{T-P}}]{\Gamma \vdash e_1~p~e_2 : int}{ 
 \Gamma \vdash e_1 : int & \Gamma \vdash e_2 : int}
\end{equation*}

\item[] 
\begin{equation*}
 \infer[\mathsf{\text{T-NIL}}]{\Gamma \vdash nil : LS}{}
\end{equation*}

\item[]
\begin{mathpar}  
  \infer[\mathsf{\text{T-CONSL}}]{\Gamma \vdash consl~e_1~e_2 : LS}{
    \Gamma \vdash e_1 : \text{int} & \Gamma \vdash e_2 : LS}
  \and
  \infer[\mathsf{\text{T-CONSS}}]{\Gamma \vdash conss~e : \text{LS}}{
    \Gamma \vdash e : unit \rightarrow LS}
\end{mathpar}

\item[]
\begin{equation*}
  \infer[\mathsf{\text{ T-CASE }}]{\Gamma \vdash \text{ case } e \text{ of nil } \Rightarrow e_1; \text{ cons-l } x \text{ } rx \Rightarrow e_2; \text{ cons-s } x \Rightarrow e_3 : t}{
    \Gamma \vdash e : \text{ LS } & \Gamma \vdash e_1 : t & \Gamma, x:\text{ int}, \text{ }rx:\text{ LS } \vdash e_2 : t & \Gamma, x:\text{ unit } \rightarrow \text{ LS } \vdash e_3 : t}
\end{equation*}
\end{itemize}


%% \newpage
\section{CBV Operational Semantics}

$\begin{array}{l c l}
  v    & ::= & () \mid n \mid lam~x:t.e \mid nil \mid consl~v_1~v_2 \mid conss~v\\[1em]
  E    & ::= & [.] \mid if0~E~e_2~e_3 \mid E~p~e_2 \mid v_1~p~E \mid E~e_2 \mid v_1~E \mid consl~E~e_2 \mid consl~v~E \mid conss~E \mid\\[1em] 
       &     &  \mid case~E~of~nil \Rightarrow e_1; consl~x~rx \Rightarrow e_2; conss~x \Rightarrow e_3\\[1em]
\end{array}$



 

\end{document}

%%%%%%%%%%%%%%%%%%%%%%%%%%%%%%%%%%%%%%%%%%%%%%%%%%%%%%%%%%%%%%%%%%%%%%%%%%%%%%%
%=============================================================================%
%%%%%%%%%%%%%%%%%%%%%%%%%%%%%%%%%%%%%%%%%%%%%%%%%%%%%%%%%%%%%%%%%%%%%%%%%%%%%%%
